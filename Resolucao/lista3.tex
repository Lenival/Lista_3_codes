\documentclass[a4paper,oneside,12pt]{article}

\usepackage[top=3.0cm,bottom=2.0cm,left=2.9cm,right=2.9cm]{geometry}
\usepackage[portuges]{babel}
\usepackage{amsmath,amsfonts,amssymb}
\usepackage{graphicx,color}
\usepackage{enumerate}
\usepackage{amsmath}
\usepackage{mathtools}
\usepackage{color}
\usepackage{pdfpages}

\begin{document}

\bfseries
\noindent
Universidade Federal do Rio Grande do Norte \\
Programa de P\'os-Gradua\c{c}\~ao em Engenharia El\'etrica e de Computa\c{c}\~ao \\
Redes Neurais (EEC1505) \\
Prof. Adri\~ao Duarte Doria Neto \\
Alunos: Jos\'e Lenival Gomes de Fran\c{c}a, Raphael Diego Comesanha e Silva, Danilo de Santana Pena.
\mdseries

\begin{center}
Lista 3 Exerc\'icios
\end{center}

\begin{enumerate}[1.]
\item A representa\c{c}\~ao de uma determinada mensagem digital tern\'aria, isto \'e formada por tr\^es bits, forma um cubo cujos v\'ertices correspondem a mesma representa\c{c}\~ao digital. Supondo que ao transmitirmos esta mensagem a mesma seja contaminada por ru\'ido formando em torno de cada v\'ertice uma nuvem esf\'erica de valores aleat\'orios. O raio da esfera corresponde ao desvio padr\~ao do sinal de ru\'ido. Solucione o problema usando m\'aquinas de vetor de suporte linear. Compare com a solu\c{c}\~ao obtida na lista 2 onde foi usada uma rede de perceptron de Rosemblat com uma camada para atuar como classificador/decodificador. Para solu\c{c}\~ao do problema defina antes um conjunto de treinamento e um conjunto de valida\c{c}\~ao. \\

RESOLU\c{C}\~AO: \\

Inicialmente define-se a representa\c{c}\~ao da mensagem digitial tern\'aria. \'E criado um cubo centrado na posi\c{c}\~ao (0, 0, 0) e com largura 2, logo os vertices s\~ao: (-1, -1, -1), (-1, -1, 1), (-1, 1, -1), (-1, 1, 1), (1, -1, -1), (1, -1, 1), (1, 1, -1) e (1, 1, 1). \\

\end{enumerate}

\end{document}